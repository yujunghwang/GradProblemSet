
\documentclass[paper=a4, fontsize=13pt]{extarticle} % A4 paper and 11pt font size
\usepackage{longtable} % Allows tables to span multiple pages (this package must be called before hyperref)
\usepackage{natbib}
\bibliographystyle{chicago}
\renewcommand{\familydefault}{\rmdefault}
\usepackage{lmodern}
\usepackage[T1]{fontenc} % Use 8-bit encoding that has 256 glyphs
\usepackage{fourier} % Use the Adobe Utopia font for the document - comment this line to return to the LaTeX default
\usepackage[english]{babel} % English language/hyphenation
\usepackage{amsmath,amsfonts,amsthm,tikz} % Math packages
\usepackage{multicol}
\usepackage{graphicx}
\usepackage{lipsum} % Used for inserting dummy 'Lorem ipsum' text into the template
\usepackage{subfigure}
\usepackage{here}
\usepackage{setspace}
\usepackage{amssymb}
\usepackage{wasysym}
\usepackage[center]{caption}
\usepackage[hidelinks]{hyperref}

\usepackage{multirow}

\usepackage{array}
\newcolumntype{L}[1]{>{\raggedright\let\newline\\\arraybackslash\hspace{0pt}}m{#1}}
\newcolumntype{C}[1]{>{\centering\let\newline\\\arraybackslash\hspace{0pt}}m{#1}}
\newcolumntype{R}[1]{>{\raggedleft\let\newline\\\arraybackslash\hspace{0pt}}m{#1}}

\usepackage{xcolor}
\hypersetup{
    colorlinks,
    linkcolor={red!50!black},
    citecolor={blue!50!black},
    urlcolor={blue!80!black}
}
\newcommand\independent{\protect\mathpalette{\protect\independenT}{\perp}}
\def\independenT#1#2{\mathrel{\rlap{$#1#2$}\mkern2mu{#1#2}}}
\newtheorem{proposition}{Proposition}
\newtheorem{mydef}{Definition}
\newtheorem{lemma}{Lemma}
\newtheorem{thm}{Theorem}
\newtheorem{corollary}{Corollary}
\newtheorem{ass}{Assumption}
\newtheorem{nota}{Notation}
\usepackage{fullpage}
\onehalfspacing
\allowdisplaybreaks

\numberwithin{equation}{section} % Number equations within sections (i.e. 1.1, 1.2, 2.1, 2.2 instead of 1, 2, 3, 4)
\numberwithin{figure}{section} % Number figures within sections (i.e. 1.1, 1.2, 2.1, 2.2 instead of 1, 2, 3, 4)
\numberwithin{table}{section} % Number tables within sections (i.e. 1.1, 1.2, 2.1, 2.2 instead of 1, 2, 3, 4)

%\newcommand\independent{\protect\mathpalette{\protect\independenT}{\perp}}
\def\independenT#1#2{\mathrel{\rlap{$#1#2$}\mkern2mu{#1#2}}}
\newcommand{\btz}{\begin{tikzpicture}}
\newcommand{\etz}{\end{tikzpicture}}
\usetikzlibrary{snakes}

\usepackage{enumitem}
\setlist[enumerate]{itemsep=0mm}

\usepackage{pdflscape}

\usepackage{booktabs}
\usepackage{adjustbox}
\usepackage{libertine}% Linux Libertine, may favourite text font
\usepackage[euler-digits]{eulervm}% A pretty math font
\usepackage{dcolumn} % Align on the decimal point of numbers in tabular columns
     \newcolumntype{d}[1]{D{.}{.}{#1}}
\usepackage{xhfill}
\newcommand{\ditto}[1][.4pt]{\xrfill{#1}~\textquotedbl~\xrfill{#1}}
\usepackage{threeparttable} % For better formatting of table notes
%\usepackage{parskip}
\usepackage{longtable}
\usepackage{threeparttablex}
\usepackage{dcolumn}

\newcommand{\sym}[1]{\rlap{#1}}% Thanks to David Carlisle

\let\estinput=\input% define a new input command so that we can still flatten the document

\newcommand{\estwide}[3]{
		\vspace{.75ex}{
			\begin{tabular*}
			{\textwidth}{@{\hskip\tabcolsep\extracolsep\fill}l*{#2}{#3}}
			\toprule
			\estinput{#1}
			\bottomrule
			\addlinespace[.75ex]
			\end{tabular*}
			}
		}	

\newcommand{\estauto}[3]{
		\vspace{.75ex}{
			\begin{tabular}{l*{#2}{#3}}
			\toprule
			\estinput{#1}
			\bottomrule
			\addlinespace[.75ex]
			\end{tabular}
			}
		}

% Allow line breaks with \\ in specialcells
	\newcommand{\specialcell}[2][c]{%
	\begin{tabular}[#1]{@{}c@{}}#2\end{tabular}}

\newcommand{\figtext}[1]{
	\vspace{-1.9ex}
	\captionsetup{justification=justified,font=footnotesize}
	\caption*{\hspace{6pt}\hangindent=1.5em #1}
	}
\newcommand{\fignote}[1]{\figtext{\emph{Note:~}~#1}}

\newcommand{\figsource}[1]{\figtext{\emph{Source:~}~#1}}

% Add significance note with \starnote
\newcommand{\starnote}{\figtext{* p < 0.1, ** p < 0.05, *** p < 0.01. Standard errors in parentheses.}}

\usepackage{siunitx} % centering in tables
	\sisetup{
		detect-mode,
		tight-spacing	        	  = true,
		group-digits	        	  = false ,
		input-signs		              = ,
		input-symbols	 	        = ( ) [ ] - + *,
		input-open-uncertainty	= ,
		input-close-uncertainty	 = ,
		table-align-text-post	  = false
        }
\newcommand{\horrule}[1]{\rule{\linewidth}{#1}} % Create horizontal rule command with 1 argument of height
\definecolor{olivegreen}{cmyk}{0.28,0,0.5,0.68}
\usepackage{listings}
\lstset{language=R,
    basicstyle=\small\ttfamily,
    stringstyle=\color{olivegreen},
    %otherkeywords={0,1,2,3,4,5,6,7,8,9},
    morekeywords={TRUE,FALSE,fminbnd,optim,optimize,fzero},
    deletekeywords={data,frame,length,as,character},
    keywordstyle=\color{blue},
    commentstyle=\color{olivegreen},
}

\author{Yujung Hwang} % Your name
\date{\today} % Today's date or a custom date

\begin{document}

\title{	
\normalfont \normalsize 
\huge Problem Set 4
}
\author{
Instructor : Yujung Hwang \thanks{\texttt{yujungghwang@gmail.com}}} % Your name
\date{\today} % Today's date or a custom date
\maketitle % Print the title

\begin{center}
DUE DATE : 2020.9.30 time 11:00pm \\
submit your solution and code files on Blackboard page.
\end{center}

\upshape \mdseries 
\normalsize

Consider the following income process. Log wage depends on observable characteristics $Z_{it}$ and the unexplained component $u_{it}$. The unexplained income component $u_{it}$ is composed of permanent component $p_t$ which follows a random walk process, and a measurement error in reported log wage, $m_{it}$, which follows an i.i.d. normal distribution.
\begin{eqnarray}
log w_{it} &=& Z_{it}'\beta + u_{it} \\
u_{it} &=& p_{it} + m_{it}, \quad m_{it} \sim N(0,\sigma_m^2) \\
p_{it} &=& p_{it-1} + \zeta_{it}, \quad \zeta_{it} \sim N(0,\sigma_{\zeta}^2) 
\end{eqnarray}
\vspace{0.2in}
\textbf{Question 1. Income Process Moment Equation} \\
Derive moment equations to estimate the following income process parameters. \\
(a) Assume there is no selection into work. so $P_{it} =1$ for $\forall i, t$. Derive the moment equations to estimate the income process parameters $\sigma_{\zeta}^2, \sigma_{m}^2$. \\
(b) Assume there is selection into work. Set up the following auxiliary labor participation equation. The idiosyncratic preference for work $\eta_{it}$ follows a joint normal distribution with permanent component shock $\zeta_{it}$. Derive the moment equations to identify the income process parameters $\sigma_{\zeta}^2, \sigma_m^2$.
\begin{gather}
P(L_{it} =1) = P(L_{it}^*\mbox{>}0) = P(X_{it}'\gamma + \eta_{it} \mbox{>} 0) = P(\eta_{it} \mbox{>} \alpha_{it} \equiv -X_{it}'\gamma) \\
\left(
\begin{array}{c}
\zeta_{it} \\
\eta_{it}
\end{array}
\right) \sim
N(0, 
\left[
\begin{array}{c c}
\sigma_{\zeta}^2 & \sigma_{\zeta \eta} \\
\sigma_{\zeta \eta} & 1
\end{array}
\right]
)
\end{gather}
Hint : You can use the formula for truncated bivariate standard normal distribution. That is, 
\begin{gather}
\left( 
\begin{array}{c}
X1 \\
X2
\end{array}
\right) \sim N(0, \left[ \begin{array}{c c}
1 & \rho \\
\rho & 1
\end{array} \right] )
\end{gather}
then 
\begin{gather}
Var (X1 | X2) = 1-\rho^2 \\
E (X1| X2) = \rho X2 \\
E (X2^2 | X2 \mbox{>} c) = c\frac{\phi(c)}{(1-\Phi(c))} +1.
\end{gather}

\vspace{0.2in}
\vspace{0.2in}
\textbf{Question 2. Estimate Income Process Using Nonlinear Least Squares} \\
Download the dataset Pset4data.csv for this question. The dataset includes a balanced panel of 3000 individuals, including id, age, education, work status, instrumental variable for work, log wage. \\
Use the following participation equation.
\begin{gather}
P(L_{it}=1)  = P(\Pi_{it} \gamma + \eta_{it} \mbox{>} 0) 
\end{gather}
$\Pi_{it} = \{ \mbox{Age}, \mbox{Age}^2, \mbox{i.Educ (categorical)}, \mbox{noveliv} \}$.\\
The observables for log wage includes
$Z_{it} =\{ \mbox{Age}, \mbox{Age}^2, \mbox{i.Educ (categorical)}  \}$.\\

(a) Estimate the work participation probit regression model. Estimate the inverse mill's ratio, $\lambda(\alpha_{it})$. Report the relevant section in your code here.  \\
(b) Write down a regression equation to estimate the unexplained income growth $\Delta u_{it} = \zeta_{it} + m_{it} - m_{it-1}$. Report your code executing the regression. \\
(c) Using the estimated unexplained income growth from (b), estimate the income process parameters $\sigma_{\zeta}^2, \sigma_{\zeta \eta}, \sigma_m^2$ using nonlinear least squares. Report the standard error by bootstrapping the whole estimation process 100 times.  \\


\vspace{0.3in}
\textbf{[Bonus Question, +5 pt]}. Compute the standard error for the estimate in (c) using the asymptotic variance formula of two-step M estimator.\\


\end{document}