\documentclass[paper=a4, fontsize=13pt]{extarticle} % A4 paper and 11pt font size
\usepackage{longtable} % Allows tables to span multiple pages (this package must be called before hyperref)
\usepackage{natbib}
\bibliographystyle{chicago}
\renewcommand{\familydefault}{\rmdefault}
\usepackage{lmodern}
\usepackage[T1]{fontenc} % Use 8-bit encoding that has 256 glyphs
\usepackage{fourier} % Use the Adobe Utopia font for the document - comment this line to return to the LaTeX default
\usepackage[english]{babel} % English language/hyphenation
\usepackage{amsmath,amsfonts,amsthm,tikz} % Math packages
\usepackage{multicol}
\usepackage{graphicx}
\usepackage{lipsum} % Used for inserting dummy 'Lorem ipsum' text into the template
\usepackage{subfigure}
\usepackage{here}
\usepackage{setspace}
\usepackage{amssymb}
\usepackage{wasysym}
\usepackage[center]{caption}
\usepackage[hidelinks]{hyperref}

\usepackage{multirow}

\usepackage{array}
\newcolumntype{L}[1]{>{\raggedright\let\newline\\\arraybackslash\hspace{0pt}}m{#1}}
\newcolumntype{C}[1]{>{\centering\let\newline\\\arraybackslash\hspace{0pt}}m{#1}}
\newcolumntype{R}[1]{>{\raggedleft\let\newline\\\arraybackslash\hspace{0pt}}m{#1}}

\usepackage{xcolor}
\hypersetup{
    colorlinks,
    linkcolor={red!50!black},
    citecolor={blue!50!black},
    urlcolor={blue!80!black}
}
\newcommand\independent{\protect\mathpalette{\protect\independenT}{\perp}}
\def\independenT#1#2{\mathrel{\rlap{$#1#2$}\mkern2mu{#1#2}}}
\newtheorem{proposition}{Proposition}
\newtheorem{mydef}{Definition}
\newtheorem{lemma}{Lemma}
\newtheorem{thm}{Theorem}
\newtheorem{corollary}{Corollary}
\newtheorem{ass}{Assumption}
\newtheorem{nota}{Notation}
\usepackage{fullpage}
\onehalfspacing
\allowdisplaybreaks

\numberwithin{equation}{section} % Number equations within sections (i.e. 1.1, 1.2, 2.1, 2.2 instead of 1, 2, 3, 4)
\numberwithin{figure}{section} % Number figures within sections (i.e. 1.1, 1.2, 2.1, 2.2 instead of 1, 2, 3, 4)
\numberwithin{table}{section} % Number tables within sections (i.e. 1.1, 1.2, 2.1, 2.2 instead of 1, 2, 3, 4)

%\newcommand\independent{\protect\mathpalette{\protect\independenT}{\perp}}
\def\independenT#1#2{\mathrel{\rlap{$#1#2$}\mkern2mu{#1#2}}}
\newcommand{\btz}{\begin{tikzpicture}}
\newcommand{\etz}{\end{tikzpicture}}
\usetikzlibrary{snakes}

\usepackage{enumitem}
\setlist[enumerate]{itemsep=0mm}

\usepackage{pdflscape}

\usepackage{booktabs}
\usepackage{adjustbox}
\usepackage{libertine}% Linux Libertine, may favourite text font
\usepackage[euler-digits]{eulervm}% A pretty math font
\usepackage{dcolumn} % Align on the decimal point of numbers in tabular columns
     \newcolumntype{d}[1]{D{.}{.}{#1}}
\usepackage{xhfill}
\newcommand{\ditto}[1][.4pt]{\xrfill{#1}~\textquotedbl~\xrfill{#1}}
\usepackage{threeparttable} % For better formatting of table notes
%\usepackage{parskip}
\usepackage{longtable}
\usepackage{threeparttablex}
\usepackage{dcolumn}

\newcommand{\sym}[1]{\rlap{#1}}% Thanks to David Carlisle

\let\estinput=\input% define a new input command so that we can still flatten the document

\newcommand{\estwide}[3]{
		\vspace{.75ex}{
			\begin{tabular*}
			{\textwidth}{@{\hskip\tabcolsep\extracolsep\fill}l*{#2}{#3}}
			\toprule
			\estinput{#1}
			\bottomrule
			\addlinespace[.75ex]
			\end{tabular*}
			}
		}	

\newcommand{\estauto}[3]{
		\vspace{.75ex}{
			\begin{tabular}{l*{#2}{#3}}
			\toprule
			\estinput{#1}
			\bottomrule
			\addlinespace[.75ex]
			\end{tabular}
			}
		}

% Allow line breaks with \\ in specialcells
	\newcommand{\specialcell}[2][c]{%
	\begin{tabular}[#1]{@{}c@{}}#2\end{tabular}}

\newcommand{\figtext}[1]{
	\vspace{-1.9ex}
	\captionsetup{justification=justified,font=footnotesize}
	\caption*{\hspace{6pt}\hangindent=1.5em #1}
	}
\newcommand{\fignote}[1]{\figtext{\emph{Note:~}~#1}}

\newcommand{\figsource}[1]{\figtext{\emph{Source:~}~#1}}

% Add significance note with \starnote
\newcommand{\starnote}{\figtext{* p < 0.1, ** p < 0.05, *** p < 0.01. Standard errors in parentheses.}}

\usepackage{siunitx} % centering in tables
	\sisetup{
		detect-mode,
		tight-spacing	        	  = true,
		group-digits	        	  = false ,
		input-signs		              = ,
		input-symbols	 	        = ( ) [ ] - + *,
		input-open-uncertainty	= ,
		input-close-uncertainty	 = ,
		table-align-text-post	  = false
        }
\newcommand{\horrule}[1]{\rule{\linewidth}{#1}} % Create horizontal rule command with 1 argument of height
\definecolor{olivegreen}{cmyk}{0.28,0,0.5,0.68}
\usepackage{listings}
\lstset{language=R,
    basicstyle=\small\ttfamily,
    stringstyle=\color{olivegreen},
    %otherkeywords={0,1,2,3,4,5,6,7,8,9},
    morekeywords={TRUE,FALSE,fminbnd,optim,optimize,fzero},
    deletekeywords={data,frame,length,as,character},
    keywordstyle=\color{blue},
    commentstyle=\color{olivegreen},
}

\author{Yujung G. Hwang} % Your name
\date{\today} % Today's date or a custom date

\begin{document}

\title{	
\normalfont \normalsize 
\huge Problem Set 2
}
\author{
Instructor : Yujung Hwang \thanks{\texttt{yujungghwang@gmail.com}}} % Your name
\date{\today} % Today's date or a custom date
\maketitle % Print the title

\begin{center}
DUE DATE : 2020.9.17 time 11pm \\
submit your solution and code files on Blackboard
\end{center}

\upshape \mdseries 
\normalsize

\textbf{Question 1. discretizing the AR(1) income process} \\
Using the parameters below, discretize the income grid and compute the numerical approximation of transition probability matrix. Every year, an agent receives an income flow $y_t$. The log of $y_t$ follows an AR (1) process with mean zero and an autocorrelation $\rho$. 
\begin{gather}
log y_{t} = \rho log y_{t-1} + \nu_t, \quad \nu_t \sim N(0, \sigma_{\nu}^2)
\end{gather}
\begin{table}[H]
\centering
\begin{tabular}{c l c}
\hline
Parameter & Description & Value \\
\hline
$N_y$ & number of Income Grid & 5 \\
$\rho$ & AR(1) process coefficient & 0.75 \\
$\sigma_{\nu}$ & SD of shock & 0.25 \\
\hline
\end{tabular}
\end{table}

\vspace{0.2in}
\textbf{Question 2. cake-eating problem with uncertain income} \\
In this question, you are asked to solve for the policy function (consumption plan) $c_t(a_t)$ for the cake-eating problem with income uncertainty using various solution methods. You should use the discretized income process in Question 1. Borrowing/saving is ALLOWED. \\
The flow utility function is CRRA.
\begin{gather}
u(c) = \frac{c^{1-\gamma}}{(1-\gamma)}
\end{gather}
The structural parameter values are as follows : \\
\begin{table}[H]
\centering
\begin{tabular}{c l c}
\hline
Parameter & Description & Value \\
\hline
$T$ & Lifetime & 40 \\
$r$ & interest rate & 0.03 \\
$\beta$ & time discount & 0.95 \\
$\gamma$ & CRRA parameter & 1.5 \\
$a_0$ & initial asset & 0 \\
$\underline{c}$ & minimum consumption & $10^-5$ \\
\hline
\end{tabular}
\end{table}
When you set an asset grid, use age-specific grid and use an unequal grid point generated from log transformation and use the $N_A=20$ gridpoints. \\
When you need to interpolate any function, use "linear" interpolation.\\
(a) Compute $c_t(a_t)$ using value function. \\
(b) Compute $c_t(a_t)$ using Euler equation without linear transformation of marginal utility function. \\
(c) Compute $c_t(a_t)$ using Euler equation with linear transformation of marginal utility function. \\

Show in one graph the solution (a) - (c) for age t = 20 and the highest income and the lowest income. Discuss the differences. \\



\vspace{0.2in}
\textbf{Question 3. computing simulated moments on consumption and asset} \\
Using the policy function solution (c) computed in Question 2, simulate the consumption decision for 100 individuals and plot the mean asset and mean consumption over the life-cycle. Comment on the mean asset and the mean consumption path using the relative size of $\beta$ and r. \\
\color{red} Remember to set seed for your code. Otherwise, your result will not be replicable.\\


\end{document}