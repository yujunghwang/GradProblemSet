\documentclass[paper=a4, fontsize=13pt]{extarticle} % A4 paper and 11pt font size
\usepackage{longtable} % Allows tables to span multiple pages (this package must be called before hyperref)
\usepackage{natbib}
\bibliographystyle{chicago}
\renewcommand{\familydefault}{\rmdefault}
\usepackage{lmodern}
\usepackage[T1]{fontenc} % Use 8-bit encoding that has 256 glyphs
\usepackage{fourier} % Use the Adobe Utopia font for the document - comment this line to return to the LaTeX default
\usepackage[english]{babel} % English language/hyphenation
\usepackage{amsmath,amsfonts,amsthm,tikz} % Math packages
\usepackage{multicol}
\usepackage{graphicx}
\usepackage{lipsum} % Used for inserting dummy 'Lorem ipsum' text into the template
\usepackage{subfigure}
\usepackage{here}
\usepackage{setspace}
\usepackage{amssymb}
\usepackage{wasysym}
\usepackage[center]{caption}
\usepackage[hidelinks]{hyperref}

\usepackage{multirow}

\usepackage{array}
\newcolumntype{L}[1]{>{\raggedright\let\newline\\\arraybackslash\hspace{0pt}}m{#1}}
\newcolumntype{C}[1]{>{\centering\let\newline\\\arraybackslash\hspace{0pt}}m{#1}}
\newcolumntype{R}[1]{>{\raggedleft\let\newline\\\arraybackslash\hspace{0pt}}m{#1}}

\usepackage{xcolor}
\hypersetup{
    colorlinks,
    linkcolor={red!50!black},
    citecolor={blue!50!black},
    urlcolor={blue!80!black}
}
\newcommand\independent{\protect\mathpalette{\protect\independenT}{\perp}}
\def\independenT#1#2{\mathrel{\rlap{$#1#2$}\mkern2mu{#1#2}}}
\newtheorem{proposition}{Proposition}
\newtheorem{mydef}{Definition}
\newtheorem{lemma}{Lemma}
\newtheorem{thm}{Theorem}
\newtheorem{corollary}{Corollary}
\newtheorem{ass}{Assumption}
\newtheorem{nota}{Notation}
\usepackage{fullpage}
\onehalfspacing
\allowdisplaybreaks

\numberwithin{equation}{section} % Number equations within sections (i.e. 1.1, 1.2, 2.1, 2.2 instead of 1, 2, 3, 4)
\numberwithin{figure}{section} % Number figures within sections (i.e. 1.1, 1.2, 2.1, 2.2 instead of 1, 2, 3, 4)
\numberwithin{table}{section} % Number tables within sections (i.e. 1.1, 1.2, 2.1, 2.2 instead of 1, 2, 3, 4)

%\newcommand\independent{\protect\mathpalette{\protect\independenT}{\perp}}
\def\independenT#1#2{\mathrel{\rlap{$#1#2$}\mkern2mu{#1#2}}}
\newcommand{\btz}{\begin{tikzpicture}}
\newcommand{\etz}{\end{tikzpicture}}
\usetikzlibrary{snakes}

\usepackage{enumitem}
\setlist[enumerate]{itemsep=0mm}

\usepackage{pdflscape}

\usepackage{booktabs}
\usepackage{adjustbox}
\usepackage{libertine}% Linux Libertine, may favourite text font
\usepackage[euler-digits]{eulervm}% A pretty math font
\usepackage{dcolumn} % Align on the decimal point of numbers in tabular columns
     \newcolumntype{d}[1]{D{.}{.}{#1}}
\usepackage{xhfill}
\newcommand{\ditto}[1][.4pt]{\xrfill{#1}~\textquotedbl~\xrfill{#1}}
\usepackage{threeparttable} % For better formatting of table notes
%\usepackage{parskip}
\usepackage{longtable}
\usepackage{threeparttablex}
\usepackage{dcolumn}

\newcommand{\sym}[1]{\rlap{#1}}% Thanks to David Carlisle

\let\estinput=\input% define a new input command so that we can still flatten the document

\newcommand{\estwide}[3]{
		\vspace{.75ex}{
			\begin{tabular*}
			{\textwidth}{@{\hskip\tabcolsep\extracolsep\fill}l*{#2}{#3}}
			\toprule
			\estinput{#1}
			\bottomrule
			\addlinespace[.75ex]
			\end{tabular*}
			}
		}	

\newcommand{\estauto}[3]{
		\vspace{.75ex}{
			\begin{tabular}{l*{#2}{#3}}
			\toprule
			\estinput{#1}
			\bottomrule
			\addlinespace[.75ex]
			\end{tabular}
			}
		}

% Allow line breaks with \\ in specialcells
	\newcommand{\specialcell}[2][c]{%
	\begin{tabular}[#1]{@{}c@{}}#2\end{tabular}}

\newcommand{\figtext}[1]{
	\vspace{-1.9ex}
	\captionsetup{justification=justified,font=footnotesize}
	\caption*{\hspace{6pt}\hangindent=1.5em #1}
	}
\newcommand{\fignote}[1]{\figtext{\emph{Note:~}~#1}}

\newcommand{\figsource}[1]{\figtext{\emph{Source:~}~#1}}

% Add significance note with \starnote
\newcommand{\starnote}{\figtext{* p < 0.1, ** p < 0.05, *** p < 0.01. Standard errors in parentheses.}}

\usepackage{siunitx} % centering in tables
	\sisetup{
		detect-mode,
		tight-spacing	        	  = true,
		group-digits	        	  = false ,
		input-signs		              = ,
		input-symbols	 	        = ( ) [ ] - + *,
		input-open-uncertainty	= ,
		input-close-uncertainty	 = ,
		table-align-text-post	  = false
        }
\newcommand{\horrule}[1]{\rule{\linewidth}{#1}} % Create horizontal rule command with 1 argument of height

\author{Yujung Hwang} % Your name
\date{September 3, 2020} % Today's date or a custom date

\begin{document}

\title{	
\normalfont \normalsize 
\huge Problem Set 1
}
\author{
Instructor : Yujung Hwang \thanks{\texttt{yujungghwang@gmail.com}}} % Your name
\date{\today} % Today's date or a custom date
\maketitle % Print the title

\upshape \mdseries 
\begin{center}
DUE DATE : 2020.9.9. time 11:00pm  \\
submit your solution and code files on Blackboard page.
\end{center}

\normalsize
\textbf{Question 1. Analytical solution for the cake-eating problem with no income flows} \\
An agent starts life with an asset $a_1$ and live for T periods. Saving is allowed. Flow utility function is the CRRA utility function. Derive an equation for the optimal consumption plan. (Derive equation (2) in slide Lecture 1, page 14) \\

\textbf{Question 2. Understanding numerical errors} \\
\textit{Note : this question was intended to use an exogenous grid for asset. you may compare solutions using exogenous grids and an endogenous grid in the Bonus question below.} \\
In this question, you are asked to solve for the policy function (consumption plan) $c_t(a_t)$ for the cake-eating problem with no income flow using various solution methods. \\
The flow utility function is CRRA.
\begin{gather}
u(c) = \frac{c^{1-\gamma}}{(1-\gamma)}
\end{gather}
The structural parameter values are as follows : \\
\begin{table}[H]
\centering
\begin{tabular}{c l c}
\hline
Parameter & Description & Value \\
\hline
$T$ & Lifetime & 40 \\
$r$ & interest rate & 0.03 \\
$\beta$ & time discount & 0.95 \\
$\gamma$ & CRRA parameter & 1.5 \\
$a_0$ & initial asset & 1 \\
$\underline{c}$ & minimum consumption & $10^-5$ \\
\hline
\end{tabular}
\end{table}
When you set an asset grid, use age-specific grid and use an unequal grid point generated from log transformation and use the $N_A=20$ gridpoints. \\
When you need to interpolate any function, use "linear" interpolation.\\
(a) Compute $c_t(a_t)$ using value function. \\
(b) Compute $c_t(a_t)$ using Euler equation without linear transformation of marginal utility function. \\
(c) Compute $c_t(a_t)$ using Euler equation with linear transformation of marginal utility function. \\

Show in one graph the true analytical solution for $t=20$ derived in Question 1 and the solution (a) - (c) for age t=20. Discuss the differences. \\

\textbf{Question 3. Simulating consumption and asset path using the solution} \\
Using the policy function solution (c) computed in Question 2, simulate the consumption and saving decision and plot them in the same graph. \\


\vspace{0.2in}
\textbf{[Bonus Question] You are not required to do this, but if you successfully solve this question, you get extra bonus points [+5 pts in your final grade].} \\
Compare the computation time to solve Question 2 using exogenous grid points (graph (a), (b), (c) respectively) and using endogenous grid point. Add a policy function $c_t(a_t)$ computed from endogenous grid point method and discuss the differences from solutions for Question 2. \\



\end{document}