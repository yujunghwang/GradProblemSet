\documentclass[paper=a4, fontsize=13pt]{extarticle} % A4 paper and 11pt font size
\usepackage{longtable} % Allows tables to span multiple pages (this package must be called before hyperref)
\usepackage{natbib}
\bibliographystyle{chicago}
\renewcommand{\familydefault}{\rmdefault}
\usepackage{lmodern}
\usepackage[T1]{fontenc} % Use 8-bit encoding that has 256 glyphs
\usepackage{fourier} % Use the Adobe Utopia font for the document - comment this line to return to the LaTeX default
\usepackage[english]{babel} % English language/hyphenation
\usepackage{amsmath,amsfonts,amsthm,tikz} % Math packages
\usepackage{multicol}
\usepackage{graphicx}
\usepackage{lipsum} % Used for inserting dummy 'Lorem ipsum' text into the template
\usepackage{subfigure}
\usepackage{here}
\usepackage{setspace}
\usepackage{amssymb}
\usepackage{wasysym}
\usepackage[center]{caption}
\usepackage[hidelinks]{hyperref}

\usepackage{multirow}

\usepackage{array}
\newcolumntype{L}[1]{>{\raggedright\let\newline\\\arraybackslash\hspace{0pt}}m{#1}}
\newcolumntype{C}[1]{>{\centering\let\newline\\\arraybackslash\hspace{0pt}}m{#1}}
\newcolumntype{R}[1]{>{\raggedleft\let\newline\\\arraybackslash\hspace{0pt}}m{#1}}

\usepackage{xcolor}
\hypersetup{
    colorlinks,
    linkcolor={red!50!black},
    citecolor={blue!50!black},
    urlcolor={blue!80!black}
}
\newcommand\independent{\protect\mathpalette{\protect\independenT}{\perp}}
\def\independenT#1#2{\mathrel{\rlap{$#1#2$}\mkern2mu{#1#2}}}
\newtheorem{proposition}{Proposition}
\newtheorem{mydef}{Definition}
\newtheorem{lemma}{Lemma}
\newtheorem{thm}{Theorem}
\newtheorem{corollary}{Corollary}
\newtheorem{ass}{Assumption}
\newtheorem{nota}{Notation}
\usepackage{fullpage}
\onehalfspacing
\allowdisplaybreaks

\numberwithin{equation}{section} % Number equations within sections (i.e. 1.1, 1.2, 2.1, 2.2 instead of 1, 2, 3, 4)
\numberwithin{figure}{section} % Number figures within sections (i.e. 1.1, 1.2, 2.1, 2.2 instead of 1, 2, 3, 4)
\numberwithin{table}{section} % Number tables within sections (i.e. 1.1, 1.2, 2.1, 2.2 instead of 1, 2, 3, 4)

%\newcommand\independent{\protect\mathpalette{\protect\independenT}{\perp}}
\def\independenT#1#2{\mathrel{\rlap{$#1#2$}\mkern2mu{#1#2}}}
\newcommand{\btz}{\begin{tikzpicture}}
\newcommand{\etz}{\end{tikzpicture}}
\usetikzlibrary{snakes}

\usepackage{enumitem}
\setlist[enumerate]{itemsep=0mm}

\usepackage{pdflscape}

\usepackage{booktabs}
\usepackage{adjustbox}
\usepackage{libertine}% Linux Libertine, may favourite text font
\usepackage[euler-digits]{eulervm}% A pretty math font
\usepackage{dcolumn} % Align on the decimal point of numbers in tabular columns
     \newcolumntype{d}[1]{D{.}{.}{#1}}
\usepackage{xhfill}
\newcommand{\ditto}[1][.4pt]{\xrfill{#1}~\textquotedbl~\xrfill{#1}}
\usepackage{threeparttable} % For better formatting of table notes
%\usepackage{parskip}
\usepackage{longtable}
\usepackage{threeparttablex}
\usepackage{dcolumn}

\newcommand{\sym}[1]{\rlap{#1}}% Thanks to David Carlisle

\let\estinput=\input% define a new input command so that we can still flatten the document

\newcommand{\estwide}[3]{
		\vspace{.75ex}{
			\begin{tabular*}
			{\textwidth}{@{\hskip\tabcolsep\extracolsep\fill}l*{#2}{#3}}
			\toprule
			\estinput{#1}
			\bottomrule
			\addlinespace[.75ex]
			\end{tabular*}
			}
		}	

\newcommand{\estauto}[3]{
		\vspace{.75ex}{
			\begin{tabular}{l*{#2}{#3}}
			\toprule
			\estinput{#1}
			\bottomrule
			\addlinespace[.75ex]
			\end{tabular}
			}
		}

% Allow line breaks with \\ in specialcells
	\newcommand{\specialcell}[2][c]{%
	\begin{tabular}[#1]{@{}c@{}}#2\end{tabular}}

\newcommand{\figtext}[1]{
	\vspace{-1.9ex}
	\captionsetup{justification=justified,font=footnotesize}
	\caption*{\hspace{6pt}\hangindent=1.5em #1}
	}
\newcommand{\fignote}[1]{\figtext{\emph{Note:~}~#1}}

\newcommand{\figsource}[1]{\figtext{\emph{Source:~}~#1}}

% Add significance note with \starnote
\newcommand{\starnote}{\figtext{* p < 0.1, ** p < 0.05, *** p < 0.01. Standard errors in parentheses.}}

\usepackage{siunitx} % centering in tables
	\sisetup{
		detect-mode,
		tight-spacing	        	  = true,
		group-digits	        	  = false ,
		input-signs		              = ,
		input-symbols	 	        = ( ) [ ] - + *,
		input-open-uncertainty	= ,
		input-close-uncertainty	 = ,
		table-align-text-post	  = false
        }
\newcommand{\horrule}[1]{\rule{\linewidth}{#1}} % Create horizontal rule command with 1 argument of height
\definecolor{olivegreen}{cmyk}{0.28,0,0.5,0.68}
\usepackage{listings}
\lstset{language=R,
    basicstyle=\small\ttfamily,
    stringstyle=\color{olivegreen},
    %otherkeywords={0,1,2,3,4,5,6,7,8,9},
    morekeywords={TRUE,FALSE,fminbnd,optim,optimize,fzero},
    deletekeywords={data,frame,length,as,character},
    keywordstyle=\color{blue},
    commentstyle=\color{olivegreen},
}

\author{Yujung G. Hwang} % Your name
\date{\today} % Today's date or a custom date

\begin{document}

\title{	
\normalfont \normalsize 
\huge Problem Set 4 Solution
}
\author{
Instructor : Yujung Hwang \thanks{\texttt{yujungghwang@gmail.com}}} % Your name
\date{\today} % Today's date or a custom date
\maketitle % Print the title

\upshape \mdseries 
\normalsize

\textbf{Question 1.} \\
(a) 
\begin{eqnarray*}
E \left[ \Delta u_{it} (\Delta u_{it-1} + \Delta u_{it} + \Delta u_{it+1}) \right] &=& E \left[ (\zeta_{it} + m_{it}) ( \zeta_{it-1} + \zeta_{it} + \zeta_{it+1} - m_{it-1} + m_{it+1} ) \right] = \sigma_{\zeta}^2 \\
E \left[ \Delta u_{it} \Delta u_{it-1} \right] &=& E \left[ (\zeta_{it} + m_{it} - m_{it-1}) (\zeta_{it-1} + m_{it-1} - m_{it-2}) \right] = -\sigma_{m}^2 
\end{eqnarray*}
(b)
Note that $\alpha_{it}$ is a random variable.
\begin{eqnarray*}
E \left[ \Delta u_{it} | L_{it}=1, L_{it-1}=1 \right] &=& \sigma_{\zeta \eta} E \left[ \frac{\phi(\alpha_{it})}{1-\Phi(\alpha_{it})} \right] \\
E \left[ \Delta u_{it} (\Delta u_{it-1} + \Delta u_{it} + \Delta u_{it+1}) | L_{it-2} =1, L_{it-1}=1, L_{it}=1, L_{it+1}=1  \right] &=& \sigma_{\zeta}^2 + \sigma_{\zeta \eta}^2 E\left[ \frac{\phi (\alpha_{it})}{1-\Phi (\alpha_{it})} \alpha_{it}  \right] \\
E \left[ \Delta u_{it}^2 | L_{it}=1, L_{it-1}=1 \right] &=&  \sigma_{\zeta}^2 + \sigma_{\zeta \eta}^2 E\left[ \frac{\phi (\alpha_{it})}{1-\Phi (\alpha_{it})} \alpha_{it} \right] + 2 \sigma_m^2  \end{eqnarray*}
To see this, 
\begin{eqnarray*}
E \left[ \Delta u_{it} | L_{it}=1, L_{it-1}=1 \right] &=& E\left[ E \left[ \zeta_{it} + m_{it} - m_{it-1} | \eta_{it} \mbox{>} \alpha_{it}, \eta_{it-1} \mbox{>} \alpha_{it-1} \right] \right] \\
&=& E\left[ E \left[ \zeta_{it} | \eta_{it} \mbox{>} \alpha_{it} \right] \right] \\
&=& \sigma_{\zeta \eta} E \left[ \frac{\phi(\alpha_{it})}{1-\Phi(\alpha_{it})} \right] (\because \mbox{truncated bivariate normal distribution formula})
\end{eqnarray*}
\begin{eqnarray*}
&& E \left[ \Delta u_{it} (\Delta u_{it-1} + \Delta u_{it} + \Delta u_{it+1}) | L_{it-2} =1, L_{it-1}=1, L_{it}=1, L_{it+1}=1  \right] \\
&& = E\left[ E \left[ \zeta_{it}^2 | \eta_{it} \mbox{>} \alpha_{it} \right] \right] \\
&& = \sigma_{\zeta}^2 E\left[ E \left[ \tilde{\zeta}_{it}^2 | \eta_{it} \mbox{>} \alpha_{it} \right] \right] \quad  (\because \tilde{\zeta}_{it} \equiv \zeta_{it} / \sigma_{\zeta}) \\
&& = \sigma_{\zeta}^2 E\left[ E \left[ E \left[ \tilde{\zeta}_{it}^2 | \eta_{it} \right] | \eta_{it} \mbox{>} \alpha_{it} \right] \right] \\
&& = \sigma_{\zeta}^2 E \left[ E \left[ Var(\tilde{\zeta}_{it} | \eta_{it}) + E \left[ \tilde{\zeta}_{it} | \eta_{it} \right]^2  | \eta_{it} \mbox{>} \alpha_{it} \right] \right] \\
&& = \sigma_{\zeta}^2 E\left[ E \left[ 1 - \rho_{\zeta \eta}^2 +  \rho_{\zeta \eta}^2 \eta_{it}^2 | \eta_{it} \mbox{>} \alpha_{it} \right] \right]  \\
&& = \sigma_{\zeta}^2 \left( 1 - \rho_{\zeta \eta}^2 + \rho_{\zeta \eta}^2 E\left[ E \left[ \eta_{it}^2 | \eta_{it} \mbox{>} \alpha_{it}  \right] \right]  \right) \\
&& = \sigma_{\zeta}^2 ( 1 - \rho_{\zeta \eta}^2 + \rho_{\zeta \eta}^2  (  E\left[ \alpha_{it} \frac{\phi(\alpha_{it})}{1-\Phi(\alpha_{it})} \right] + 1) ) \\
&& = \sigma_{\zeta}^2 ( 1 + \rho_{\zeta \eta}^2 E\left[ \alpha_{it} \frac{\phi(\alpha_{it})}{1 - \Phi(\alpha_{it}) } \right] )
\end{eqnarray*}
\begin{eqnarray*}
&& E \left[ \Delta u_{it}^2 | L_{it}=1, L_{it-1}=1 \right] \\
&& = E\left[ E \left[ (\zeta_{it} + m_{it} - m_{it-1})^2 | \eta_{it} \mbox{>} \alpha_{it}, \eta_{it-1} \mbox{>} \alpha_{it-1} \right] \right] \\
&& = E\left[ E \left[ \zeta_{it}^2 | \eta_{it} \mbox{>} \alpha_{it} \right] \right] + 2 \sigma_m^2 \\
&& = \sigma_{\zeta}^2 ( 1 + \rho_{\zeta \eta}^2 E\left[ \alpha_{it} \frac{\phi(\alpha_{it})}{1 - \Phi(\alpha_{it}) } \right]  ) + 2 \sigma_m^2
\end{eqnarray*}


\vspace{0.2in}
\textbf{Question 2.} \\
(a) \\
\begin{lstlisting}
  workeq <- glm(work ~ age + age2 + factor(educ) + noveliv, family = binomial(link = "probit"), data = data)
  # (a)
  alphahat <- -cbind(rep(1,nobs),data$age,data$age^2,(data$educ=="College Graduate"),
                     (data$educ=="Postgraduate"),(data$educ=="Some College"),
                     data$noveliv)%*%workeq$coefficients
  lambdahat <- dnorm(alphahat)/(1-pnorm(alphahat))
\end{lstlisting}
(b) 
\begin{gather}
log w_{it} - log w_{it-1} = \Delta Z_{it}' \beta + \zeta_{it} + m_{it} + m_{it-1}
\end{gather}
\begin{lstlisting}
reg1  <- lm(d1lnwage ~ d1age2 + lambdahat)
# compute residual and find uhat
duhat <- d1lnwage - reg1$coefficients[1] - reg1$coefficients[2]*d1age2
\end{lstlisting}
(c)
\begin{lstlisting}
> paramest
[1] 0.03119798 0.03159257 0.01941181
> stderr
[1] 0.0010169239 0.0125732414 0.0006523193
\end{lstlisting}

\end{document}