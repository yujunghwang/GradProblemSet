\documentclass[paper=a4, fontsize=13pt]{extarticle} % A4 paper and 11pt font size
\usepackage{longtable} % Allows tables to span multiple pages (this package must be called before hyperref)
\usepackage{natbib}
\bibliographystyle{chicago}
\renewcommand{\familydefault}{\rmdefault}
\usepackage{lmodern}
\usepackage[T1]{fontenc} % Use 8-bit encoding that has 256 glyphs
\usepackage{fourier} % Use the Adobe Utopia font for the document - comment this line to return to the LaTeX default
\usepackage[english]{babel} % English language/hyphenation
\usepackage{amsmath,amsfonts,amsthm,tikz} % Math packages
\usepackage{multicol}
\usepackage{graphicx}
\usepackage{lipsum} % Used for inserting dummy 'Lorem ipsum' text into the template
\usepackage{subfigure}
\usepackage{here}
\usepackage{setspace}
\usepackage{amssymb}
\usepackage{wasysym}
\usepackage[center]{caption}
\usepackage[hidelinks]{hyperref}

\usepackage{multirow}

\usepackage{array}
\newcolumntype{L}[1]{>{\raggedright\let\newline\\\arraybackslash\hspace{0pt}}m{#1}}
\newcolumntype{C}[1]{>{\centering\let\newline\\\arraybackslash\hspace{0pt}}m{#1}}
\newcolumntype{R}[1]{>{\raggedleft\let\newline\\\arraybackslash\hspace{0pt}}m{#1}}

\usepackage{xcolor}
\hypersetup{
    colorlinks,
    linkcolor={red!50!black},
    citecolor={blue!50!black},
    urlcolor={blue!80!black}
}
\newcommand\independent{\protect\mathpalette{\protect\independenT}{\perp}}
\def\independenT#1#2{\mathrel{\rlap{$#1#2$}\mkern2mu{#1#2}}}
\newtheorem{proposition}{Proposition}
\newtheorem{mydef}{Definition}
\newtheorem{lemma}{Lemma}
\newtheorem{thm}{Theorem}
\newtheorem{corollary}{Corollary}
\newtheorem{ass}{Assumption}
\newtheorem{nota}{Notation}
\usepackage{fullpage}
\onehalfspacing
\allowdisplaybreaks

\numberwithin{equation}{section} % Number equations within sections (i.e. 1.1, 1.2, 2.1, 2.2 instead of 1, 2, 3, 4)
\numberwithin{figure}{section} % Number figures within sections (i.e. 1.1, 1.2, 2.1, 2.2 instead of 1, 2, 3, 4)
\numberwithin{table}{section} % Number tables within sections (i.e. 1.1, 1.2, 2.1, 2.2 instead of 1, 2, 3, 4)

%\newcommand\independent{\protect\mathpalette{\protect\independenT}{\perp}}
\def\independenT#1#2{\mathrel{\rlap{$#1#2$}\mkern2mu{#1#2}}}
\newcommand{\btz}{\begin{tikzpicture}}
\newcommand{\etz}{\end{tikzpicture}}
\usetikzlibrary{snakes}

\usepackage{enumitem}
\setlist[enumerate]{itemsep=0mm}

\usepackage{pdflscape}

\usepackage{booktabs}
\usepackage{adjustbox}
\usepackage{libertine}% Linux Libertine, may favourite text font
\usepackage[euler-digits]{eulervm}% A pretty math font
\usepackage{dcolumn} % Align on the decimal point of numbers in tabular columns
     \newcolumntype{d}[1]{D{.}{.}{#1}}
\usepackage{xhfill}
\newcommand{\ditto}[1][.4pt]{\xrfill{#1}~\textquotedbl~\xrfill{#1}}
\usepackage{threeparttable} % For better formatting of table notes
%\usepackage{parskip}
\usepackage{longtable}
\usepackage{threeparttablex}
\usepackage{dcolumn}

\newcommand{\sym}[1]{\rlap{#1}}% Thanks to David Carlisle

\let\estinput=\input% define a new input command so that we can still flatten the document

\newcommand{\estwide}[3]{
		\vspace{.75ex}{
			\begin{tabular*}
			{\textwidth}{@{\hskip\tabcolsep\extracolsep\fill}l*{#2}{#3}}
			\toprule
			\estinput{#1}
			\bottomrule
			\addlinespace[.75ex]
			\end{tabular*}
			}
		}	

\newcommand{\estauto}[3]{
		\vspace{.75ex}{
			\begin{tabular}{l*{#2}{#3}}
			\toprule
			\estinput{#1}
			\bottomrule
			\addlinespace[.75ex]
			\end{tabular}
			}
		}

% Allow line breaks with \\ in specialcells
	\newcommand{\specialcell}[2][c]{%
	\begin{tabular}[#1]{@{}c@{}}#2\end{tabular}}

\newcommand{\figtext}[1]{
	\vspace{-1.9ex}
	\captionsetup{justification=justified,font=footnotesize}
	\caption*{\hspace{6pt}\hangindent=1.5em #1}
	}
\newcommand{\fignote}[1]{\figtext{\emph{Note:~}~#1}}

\newcommand{\figsource}[1]{\figtext{\emph{Source:~}~#1}}

% Add significance note with \starnote
\newcommand{\starnote}{\figtext{* p < 0.1, ** p < 0.05, *** p < 0.01. Standard errors in parentheses.}}

\usepackage{siunitx} % centering in tables
	\sisetup{
		detect-mode,
		tight-spacing	        	  = true,
		group-digits	        	  = false ,
		input-signs		              = ,
		input-symbols	 	        = ( ) [ ] - + *,
		input-open-uncertainty	= ,
		input-close-uncertainty	 = ,
		table-align-text-post	  = false
        }
\newcommand{\horrule}[1]{\rule{\linewidth}{#1}} % Create horizontal rule command with 1 argument of height

\begin{document}

\title{	
\normalfont \normalsize 
\huge Problem Set 7
}
\author{
Instructor : Yujung Hwang \thanks{\texttt{yujungghwang@gmail.com}}} % Your name
\date{\today} % Today's date or a custom date
\maketitle % Print the title

\upshape \mdseries 
\begin{center}
DUE DATE : 2020.10.21. time 11:00pm \\
submit your solution and code files on Blackboard page.
\end{center}

\textit{Note :  Solving a limited commitment and full commitment model may take longer than solving a single agent problem, which we practiced so far. When you estimate these kinds of models in practice, you should consider parallelizing your code / translating your code in another language. This pset will not ask you to estimate these models to make your task simpler.} \\

\normalsize
\textbf{Question 1 (Solving Limited Commitment and Full Commitment).} \\
(a) Solve the married couple's problem of consumption, saving, and divorce using a limited commitment model. The outside option of marriage is the value of divorce, and the value of divorce is equal to a single agent's consumption / saving problem (that is, solution of your pset 2).\\
\indent There is no public consumption, economies of scale within household. Each spouse consumes private consumption good each period. There is no preference on marriage either (so-called "love" between couples.). So the only gain from marriage is informal insurance (consumption smoothing). \\
\indent The income flow follows the AR (1) process in log. Both spouses work in the labor market and share the same income process parameters.
\begin{gather}
log y_{jt} = \rho log y_{jt-1} + \nu_{jt}, \quad \nu_{jt} \sim N(0, \sigma_{\nu}^2), \quad j=\{1,2\}
\end{gather}
\indent Each spouse member gets flow utility from private consumption only and the functional form is CRRA.
\begin{gather}
u(c_{jt}) = \frac{c_{jt}^{1-\gamma}}{(1-\gamma)}
\end{gather}
The structural parameter values are as follows : \\
\begin{table}[H]
\centering
\begin{tabular}{c l c}
\hline
Parameter & Description & Value \\
\hline
$N_y$ & number of Income Grid & 5 \\
$\rho$ & AR(1) process coefficient & 0.75 \\
$\sigma_{\nu}$ & SD of shock & 0.25 \\
$T$ & Lifetime & 10 \\
$r$ & interest rate & 0.03 \\
$\beta$ & time discount & 0.95 \\
$\gamma$ & CRRA parameter & 1.5 \\
$a_0$ & initial asset & 0 \\
$\underline{c}$ & minimum consumption & $10^-5$ \\
\hline
\end{tabular}
\end{table}
When you set an asset grid, use age-specific grid and use an unequal grid point generated from log transformation and use the $N_A=10$ gridpoints. When you need to interpolate any function, use "linear" interpolation.\\

(b) Solve the same problem as written in (a) but assume full commitment within household.\\

\vspace{0.2in}
(c) Simulate 1000 married couple's future divorce and consumption decisions using initial pareto weight (for spouse member 1) $\theta_0$ = 0.3 and the solution in (a) and (b) and the following initial parameters. Graph the average log consumption of spouse member 1 and 2 under limited commitment and full commitment. Discuss the differences.

\vspace{0.2in}
(d) Graph the variance of log consumption of spouse member 1 and 2 under limited commitment and full commitment. Discuss the differences.




\end{document}