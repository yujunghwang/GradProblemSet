\documentclass[paper=a4, fontsize=13pt]{extarticle} % A4 paper and 11pt font size
\usepackage{longtable} % Allows tables to span multiple pages (this package must be called before hyperref)
\usepackage{natbib}
\bibliographystyle{chicago}
\renewcommand{\familydefault}{\rmdefault}
\usepackage{lmodern}
\usepackage[T1]{fontenc} % Use 8-bit encoding that has 256 glyphs
\usepackage{fourier} % Use the Adobe Utopia font for the document - comment this line to return to the LaTeX default
\usepackage[english]{babel} % English language/hyphenation
\usepackage{amsmath,amsfonts,amsthm,tikz} % Math packages
\usepackage{multicol}
\usepackage{graphicx}
\usepackage{lipsum} % Used for inserting dummy 'Lorem ipsum' text into the template
\usepackage{subfigure}
\usepackage{here}
\usepackage{setspace}
\usepackage{amssymb}
\usepackage{wasysym}
\usepackage[center]{caption}
\usepackage[hidelinks]{hyperref}

\usepackage{multirow}

\usepackage{array}
\newcolumntype{L}[1]{>{\raggedright\let\newline\\\arraybackslash\hspace{0pt}}m{#1}}
\newcolumntype{C}[1]{>{\centering\let\newline\\\arraybackslash\hspace{0pt}}m{#1}}
\newcolumntype{R}[1]{>{\raggedleft\let\newline\\\arraybackslash\hspace{0pt}}m{#1}}

\usepackage{xcolor}
\hypersetup{
    colorlinks,
    linkcolor={red!50!black},
    citecolor={blue!50!black},
    urlcolor={blue!80!black}
}
\newcommand\independent{\protect\mathpalette{\protect\independenT}{\perp}}
\def\independenT#1#2{\mathrel{\rlap{$#1#2$}\mkern2mu{#1#2}}}
\newtheorem{proposition}{Proposition}
\newtheorem{mydef}{Definition}
\newtheorem{lemma}{Lemma}
\newtheorem{thm}{Theorem}
\newtheorem{corollary}{Corollary}
\newtheorem{ass}{Assumption}
\newtheorem{nota}{Notation}
\usepackage{fullpage}
\onehalfspacing
\allowdisplaybreaks

\numberwithin{equation}{section} % Number equations within sections (i.e. 1.1, 1.2, 2.1, 2.2 instead of 1, 2, 3, 4)
\numberwithin{figure}{section} % Number figures within sections (i.e. 1.1, 1.2, 2.1, 2.2 instead of 1, 2, 3, 4)
\numberwithin{table}{section} % Number tables within sections (i.e. 1.1, 1.2, 2.1, 2.2 instead of 1, 2, 3, 4)

%\newcommand\independent{\protect\mathpalette{\protect\independenT}{\perp}}
\def\independenT#1#2{\mathrel{\rlap{$#1#2$}\mkern2mu{#1#2}}}
\newcommand{\btz}{\begin{tikzpicture}}
\newcommand{\etz}{\end{tikzpicture}}
\usetikzlibrary{snakes}

\usepackage{enumitem}
\setlist[enumerate]{itemsep=0mm}

\usepackage{pdflscape}

\usepackage{booktabs}
\usepackage{adjustbox}
\usepackage{libertine}% Linux Libertine, may favourite text font
\usepackage[euler-digits]{eulervm}% A pretty math font
\usepackage{dcolumn} % Align on the decimal point of numbers in tabular columns
     \newcolumntype{d}[1]{D{.}{.}{#1}}
\usepackage{xhfill}
\newcommand{\ditto}[1][.4pt]{\xrfill{#1}~\textquotedbl~\xrfill{#1}}
\usepackage{threeparttable} % For better formatting of table notes
%\usepackage{parskip}
\usepackage{longtable}
\usepackage{threeparttablex}
\usepackage{dcolumn}

\newcommand{\sym}[1]{\rlap{#1}}% Thanks to David Carlisle

\let\estinput=\input% define a new input command so that we can still flatten the document

\newcommand{\estwide}[3]{
		\vspace{.75ex}{
			\begin{tabular*}
			{\textwidth}{@{\hskip\tabcolsep\extracolsep\fill}l*{#2}{#3}}
			\toprule
			\estinput{#1}
			\bottomrule
			\addlinespace[.75ex]
			\end{tabular*}
			}
		}	

\newcommand{\estauto}[3]{
		\vspace{.75ex}{
			\begin{tabular}{l*{#2}{#3}}
			\toprule
			\estinput{#1}
			\bottomrule
			\addlinespace[.75ex]
			\end{tabular}
			}
		}

% Allow line breaks with \\ in specialcells
	\newcommand{\specialcell}[2][c]{%
	\begin{tabular}[#1]{@{}c@{}}#2\end{tabular}}

\newcommand{\figtext}[1]{
	\vspace{-1.9ex}
	\captionsetup{justification=justified,font=footnotesize}
	\caption*{\hspace{6pt}\hangindent=1.5em #1}
	}
\newcommand{\fignote}[1]{\figtext{\emph{Note:~}~#1}}

\newcommand{\figsource}[1]{\figtext{\emph{Source:~}~#1}}

% Add significance note with \starnote
\newcommand{\starnote}{\figtext{* p < 0.1, ** p < 0.05, *** p < 0.01. Standard errors in parentheses.}}

\usepackage{siunitx} % centering in tables
	\sisetup{
		detect-mode,
		tight-spacing	        	  = true,
		group-digits	        	  = false ,
		input-signs		              = ,
		input-symbols	 	        = ( ) [ ] - + *,
		input-open-uncertainty	= ,
		input-close-uncertainty	 = ,
		table-align-text-post	  = false
        }
\newcommand{\horrule}[1]{\rule{\linewidth}{#1}} % Create horizontal rule command with 1 argument of height

\begin{document}

\title{	
\normalfont \normalsize 
\huge Problem Set 7 Solution
}
\author{
Instructor : Yujung Hwang \thanks{\texttt{yujungghwang@gmail.com}}} % Your name
\date{\today} % Today's date or a custom date
\maketitle % Print the title

\upshape \mdseries 

\textbf{Question 1.} \\

\begin{figure}
\centering
\includegraphics[width=0.7\linewidth]{AvLogCons}
\end{figure}

\begin{figure}
\centering
\includegraphics[width=0.7\linewidth]{VarLogCons}
\end{figure}

\indent Under limited commitment, each spouse member can not get lower value than their outside option (autarky). Therefore, even though the initial pareto weight $\theta_0=0.3$ is in favor of household member 2, the difference in consumption quickly disappears as income of each spouse member fluctuates and one of participation constraints binds. Unlike limited commitment, the Pareto weight under full commitment doesn't change within marriage. Therefore, the spouse member 2 gets much higher consumption throughout lifetime. \\
\indent Under limited commitment, the degree of consumption smoothing is restricted because of participation constraints. On the other hand, the degree of consumption smoothing, the informal insurance, is higher under full commitment. This can be seen from the variance of log consumption graph. The full commitment lines are below the limited commitment lines. \\
\indent Note that the consumption profiles are downward sloping in both limited commitment and full commitment because of parameters $\beta (1+r) \mbox{<}1$. \\
\indent Note also that divorce didn't happen in simulated sample, because the value of informal insurance within marriage makes the couple's Pareto frontier to lie above each single's autarky values. Therefore, they can find different bargainig weights to make both spouses better off than living in autarky. This will not be true if, for example, there is additional marital preference shock, which may make divorce a better option.\\



\end{document}


